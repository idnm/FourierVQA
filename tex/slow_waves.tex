\documentclass[12 pt]{article}
\usepackage[utf8x]{inputenc}
\usepackage{amsmath}
\usepackage{amsfonts}
\usepackage{amssymb}
\usepackage{empheq}
\usepackage[sort&compress,numbers]{natbib}
\usepackage{doi}
\usepackage{esvect}
\usepackage{cancel}
\usepackage{braket}
\usepackage{hyperref}
\hypersetup{
	colorlinks   = true, %Colours links instead of ugly boxes
	urlcolor     = blue, %Colour for external hyperlinks
	linkcolor    = blue, %Colour of internal links
	citecolor   = red %Colour of citations
}

\textheight 23cm
\textwidth 18cm
\voffset=-1.2in
\hoffset= - 0.9in         
\begin{document}
\section{Notation}
\begin{itemize}
\item
\begin{align}
	f_i(\phi)=\cos^{i-1}\phi\sin^\phi=\begin{cases} \cos\phi,\quad i=0\\\sin\phi,\quad i=1\end{cases}
\end{align}
\item
\begin{align}
	f_I(\phi)=\prod_{i\in I} f_{i}(\phi)
\end{align}
\item $n$ - number of qubits
\item $M$ - number of parameters
\end{itemize}
\section{PQC}
Parametrized circuit with $M$ parameters
\begin{align}
	U(\phi)=\sum_{I\in [0,1]^M}U_I f_I\left(\frac{\phi}{2}\right)
\end{align}
where $U_I = U(\phi=\pi I)$. E.g. for $M=2$
\begin{align}U_{00}\cos\frac{\phi_1}{2}\cos\frac{\phi_2}{2}+U_{01}\cos\frac{\phi_1}{2}\sin\frac{\phi_2}{2}+U_{10}\sin\frac{\phi_1}{2}\cos\frac{\phi_2}{2}+U_{11}\sin\frac{\phi_1}{2}\sin\frac{\phi_2}{2}
\end{align}
\section{Quadratic loss and its Fourier expansion}
Quadratic loss function
\begin{align}
	L(\phi)=\Braket{0|U(\phi)^\dagger H U(\phi)|0}
\end{align}
can be expanded as
\begin{align}
	L(\phi)=\sum_{I,J} f_I\left(\frac{\phi}{2}\right)f_J\left(\frac{\phi}{2}\right)\Braket{0|U_I^\dagger H U_J|0}=\sum_{m=0}^M L_m(\phi)
\end{align}
where
\begin{align}
&L_0(\phi) = const\\
&L_1(\phi)=\sum_{i=0}^M L_{i}(\phi_i)\\
&L_2(\phi)=\sum_{i<j=0}^M L_{ij}(\phi_i, \phi_j)\\
&L_3(\phi)=\sum_{i<j<k=0}^M L_{ijk}(\phi_i, \phi_j,\phi_k)\\
&L_4(\phi)=\dots
\end{align}
And each summand in each term is is a homogeneous trigonometric polynomial of a special form
\begin{align}
	&L(\phi_1)=A\cos\phi_1+B\sin\phi_1\\
	&L(\phi_1,\phi_2)=A\cos\phi_1\cos\phi_2+B\cos{\phi_1}\sin{\phi_2}+C\sin{\phi_1}\cos{\phi_2}+D\sin{\phi_1}\sin{\phi_2}\\
	&L(\phi_1,\phi_2,\phi_3)=A\cos{\phi_1}\cos{\phi_2}\cos{\phi_3}+\dots
\end{align}
\section{Coefficients in Fourier expansion from averages}
Note that 
\begin{align}
F_0=\Braket{L(\phi)}_\phi=\frac1{(2\pi)^M}\int_0^{2\pi}\prod_i^M d\phi_i\,\, L(\phi)=\frac1{2^M}\sum_{I\in [0, 1]^M} L(\phi=\pi I)=\frac1{2^M}\sum_{I\in [0,1]^M}\Braket{0|U_I^\dagger H U_I|0}\label{F0}
\end{align}
where to get the last expression we used $\int_0^{2\pi} \prod_i^Md\phi_i\,\, f_I(\phi)f_J(\phi)=\pi^M \delta_{IJ}$. Similarly,
\begin{align}
&F_i(\phi_i)=\Braket{L(\phi)-L_0}_{\phi\neq\phi_i}\\	&F_{ij}(\phi_i,\phi_j)=\Braket{L(\phi)-L_1(\phi)-L_0}_{\phi\neq \phi_i,\phi_j}\\
&F_{ijk}(\phi_i,\phi_j,\phi_k)=\dots
\end{align}
\section{Averages for Clifford+$\phi$ circuits}
For a general circuit computing the average requires $2^M$ evaluations of the loss function. For a class of circuits that consists only of Clifford gates and Pauli rotations computing averages can be done very efficiently.

Introduce
\begin{align}
	U_m=U(\phi_m=\pi)=C'_m P_m C''_m=P_m' C 
\end{align}
Here $P_m$ is the Pauli gate corresponding to $m$-th parameter, $C'_m$ and $C''_m$ are Clifford gates to the right and left from the Pauli rotation, $C=C'_mC''_m=U(\phi=0)$ is the total Clifford of the circuit and $P'_m$ is the Pauli gate of parameter $m$ commuted to the rightmost.

Let $P'$ be the the Pauli subgroup generated by $\mathbb{I}$ and all $P'_m$. The average \eqref{F0} can be rewritten as
\begin{align}
F_0=\frac1{2^M}\sum_{P\in P'}\Braket{0|C^\dagger PHPC|0}
\end{align}
Now let $H$ itself be a Pauli string $Q$. Then $PQP=(-)^{QP}Q$ 
\begin{align}
F_0=\frac1{2^M}\sum_{P\in P'}(-)^{PQ}\Braket{0|C^\dagger Q C|0}
\end{align}
Now, if there is an element in $P'$ such that $(-)^{PQ}=-1$ the whole sum is necessarily zero. Therefore
\begin{align}
F_0=\begin{cases}\Braket{0|C^\dagger Q C|0},\quad \text{if } [P', Q]=0\\0,\quad \text{otherwise}\end{cases}
\end{align}
\end{document}
